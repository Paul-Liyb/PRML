选题5为漫画照片与人脸照片的识别与匹配问题。该问题的本质是一个跨模态的异质人脸识别问题,即漫画和真人照片两个模态,其中的人脸是异质表现的。解决这一问题,大致分为三个步骤:人脸特征表示,解决跨模态问题和设计匹配算法。

\subsection{人脸特征表示}
由于照片和漫画中的人脸是异质表现的,传统人脸识别(基于照片的人脸识别)中的特征点定位和面部特征提取不能很好地直接运用在漫画识别中。在漫画中,除了与照片中一样对主体面部的客观表现,还加入了艺术家的主观印象和绘画风格。这些变量会对人脸特征表示造成很大挑战。
\subsection{跨模态问题}
对于漫画照片而言,会有面部外观的夸张,因此原本人脸中的特征点会被夸大并出现在不合理的位置而难以被基于照片的人脸识别定位。而漫画有多种绘画风格,导致原本照片和漫画形成的双模态问题就可能变成多模态问题。因此,需要提取一般的特征,防止过拟合单个模态的特征导致引入过多噪声。
\subsection{设计匹配算法}
利用从照片和漫画提取到的特征进行处理(分类器设计),以判断他们是否是同一个人。 
\subsection{评测标准}
需要计算找出与Probe中的图片人物身份相同的Gallery图片,返回该图片的名称作为Probe图片的匹配结果,赛方计算Rank-1准确率。其中,照片与漫画交替作为Probe与Gallery测试集。\begin{align*}
    	 Rank-1=\frac{\Sigma^{n}_{i=1}I(G_{i1}=P_i)}{n}
\end{align*}

