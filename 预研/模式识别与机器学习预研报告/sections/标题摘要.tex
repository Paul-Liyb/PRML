
\begin{center}
    \erhao \sffamily 漫画人脸照片识别

    \vspace{0.3cm}

    \xiaosihao \ttfamily 李扬波2021K8009991001,廖海川2021K8009991007,\\邵永成2021K8009991004,钟\quad  杰2021K8009991008

    % \xiaowuhao (1.作者详细单位,省市 邮编;2.作者详细单位,省市 邮编)
\end{center}

\xiaowuhao{
    \noindent \sffamily 摘要: \normalfont
    本文针对漫画照片与人脸照片的识别与匹配问题,提出了一种跨模态异质人脸识别方法。该方法包括三个主要步骤:人脸特征表示、解决跨模态问题和设计匹配算法。在处理人脸特征表示时,需要定位人脸的特征点和提取面部特征,但由于漫画和照片的异质表现,传统基于照片的人脸识别方法不适用。在处理跨模态问题时,需要提取一般的特征以防止过拟合单个模态的特征导致引入过多噪声。在设计匹配算法时,利用从照片和漫画提取到的特征进行处理,以判断他们是否是同一个人。本文调研了五种方法,包括WebCaricature、图片合成法、基于Facial landmarks的特征表示、基于重构的方法和基于深度学习的方法。最后,本文以Rank-1准确率为评估标准对方法进行了比较分析,提出了我们可能的优化方向和未来研究方向。
    % 摘要内容。概括地陈述论文研究的目的、方法、结果、结论,要求200~300字。应排除本学科领域已成为常识的内容;不要把应在引言中出现的内容写入摘要,不引用参考文献;不要对论文内容作诠释和评论。不得简单重复题名中已有的信息。用第三人称,不使用“本文”、“作者”等作为主语。使用规范化的名词术语,新术语或尚无合适的汉文术语的,可用原文或译出后加括号注明。除了无法变通之外,一般不用数学公式和化学结构式,不出现插图、表格。缩略语、略称、代号,除了相邻专业的读者也能清楚理解的以外,在首次出现时必须加括号说明。结构严谨,表达简明,语义确切。
  
%     \noindent \sffamily 关键词:\normalfont 关键词1;关键词2;关键词3;关键词4
%    }

%    \begin{center}
%     \sihao Title

%     \vspace{0.3cm}

%     \xiaosihao NAME Name$^1$,NAME Name-name$^2$

%     \xiaowuhao (1. Department, City, City Zip Code, China; 2. Department, City, City Zip Code, China)

% \end{center}
% \xiaowuhao{
%     \noindent \textbf{Abstract:}英文摘要应是中文摘要的转译,所以只要简洁、准确地逐段将文意译出即可,要求250单词左右。时态用一般过去时,采用被动语态或原型动词开头。避免用阿拉伯数字作首词,不出现缩写。尽量使用短句。.

%     \noindent \textbf{Keywords: }keyword 1; keyword 2; keyword 3; keyword 4
%    }

